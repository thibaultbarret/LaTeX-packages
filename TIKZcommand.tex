\newcommand{\centeredSubsetACS}[4]{
    % #1 x coordinate of the subset 
    % #2 y coordinate of the subset 
    % #3 size of the subset
    % #4 draw style
    
    \pgfmathsetmacro{\halfSize}{(#3-1)/2}
    \draw [#4] ({#1-\halfSize}, {#2-\halfSize}) rectangle ({#1+\halfSize}, {#2+\halfSize}) ; 
    \draw [#4] ({#1-2}, {#2}) -- ({#1+2}, {#2}) ; 
    \draw [#4] ({#1}, {#2-2}) -- ({#1}, {#2+2}) ; 
}

\makeatletter
\newcommand{\gettikzxy}[3]{%
  \tikz@scan@one@point\pgfutil@firstofone#1\relax
  \edef#2{\the\pgf@x}%
  \edef#3{\the\pgf@y}%
}

\newcommand{\drawDashedCube}[2]{
    % Back part : 
    \draw[#1] (0., 0., 0.) --
        +(0., #2, 0.) --
        +(0., #2, #2) --
        +(0., 0., #2) -- cycle ; 
    
    % Front part
    \draw [#1] ({#2}, {0.}, {0.}) --
        +(0., #2, 0.) --
        +(0., #2, #2) --
        +(0., 0., #2) -- cycle ; 

    % Link rectangles :
    \draw [#1] (0., 0., 0.) -- +(#2, 0., 0.) ; 
    \draw [#1] (0., #2, 0.) -- +(#2, 0., 0.) ; 
    \draw [#1] (0., 0., #2) -- +(#2, 0., 0.) ; 
    \draw [#1] (0., #2, #2) -- +(#2, 0., 0.) ; 
}


\newcommand{\RotatedPointXYZ}[7]{%
    % #1 = alpha (X), #2 = beta (Y), #3 = gamma (Z)
    % #4 = x, #5 = y, #6 = z, #7 = nom du point (P => \Px, \Py, \Pz)

    \pgfmathsetmacro{\alpha}{#1}
    \pgfmathsetmacro{\beta}{#2}
    \pgfmathsetmacro{\gamma}{#3}
    \pgfmathsetmacro{\x}{#4}
    \pgfmathsetmacro{\y}{#5}
    \pgfmathsetmacro{\z}{#6}

    % Calcul des cosinus/sinus
    \pgfmathsetmacro{\ca}{cos(\alpha)}
    \pgfmathsetmacro{\sa}{sin(\alpha)}
    \pgfmathsetmacro{\cb}{cos(\beta)}
    \pgfmathsetmacro{\sb}{sin(\beta)}
    \pgfmathsetmacro{\cg}{cos(\gamma)}
    \pgfmathsetmacro{\sg}{sin(\gamma)}

    % Rotation XYZ : R = Rz * Ry * Rx
    \pgfmathsetmacro{\xp}{
        \cb*\cg*\x + (-\cg*\sa*\sb + \ca*\sg)*\y + (\sa*\sg + \ca*\cg*\sb)*\z
    }
    \pgfmathsetmacro{\yp}{
        -\cb*\sg*\x + (\ca*\cg + \sa*\sb*\sg)*\y + (-\cg*\sa + \ca*\sb*\sg)*\z
    }
    \pgfmathsetmacro{\zp}{
        \sb*\x + \cb*\sa*\y + \ca*\cb*\z
    }

    % Enregistrer dans \<P>x, \y, \z
    \global\expandafter\edef\csname #7x\endcsname{\xp}
    \global\expandafter\edef\csname #7y\endcsname{\yp}
    \global\expandafter\edef\csname #7z\endcsname{\zp}
}

\newcommand{\marker}[2]{%
  \tikz[baseline=-0.6ex, x=1ex, y=1ex]%
    \draw[mark=#1, mark size=1.5pt, #2] plot coordinates {(0,0)};%
}

\newcommand{\drawSubset}[4]{
    \pgfmathsetmacro{\halfSubset}{div(#3,2)}
     \draw [#4] (axis cs:{#1 -\halfSubset},{#2 - \halfSubset}) rectangle (axis cs:{#1 +\halfSubset},{#2 + \halfSubset}) ; 
}

\newcommand{\drawVSG}[4]{
    \pgfmathsetmacro{\halfVSG}{div(#3,2)}
     \draw [#4] (axis cs:{#1 -\halfVSG},{#2 - \halfVSG}) rectangle (axis cs:{#1 +\halfVSG},{#2 + \halfVSG}) ; 
}

\newcommand{\drawNotched}[1]{\begin{tikzpicture}[baseline=(current bounding box.south), scale = 0.1]
        \draw [#1] (-2., -1) arc (-90:90:1) --
        (-2, 2.5) --
        (2, 2.5) --
        (2, 1) arc (90:270:1) -- 
        (2, -2.5) -- 
        (-2, -2.5) -- 
        cycle ;
    \end{tikzpicture}
}

\newcommand{\drawNotchedExternalize}[2]{\tikzsetnextfilename{#1}\begin{tikzpicture}[baseline=(current bounding box.south), scale = 0.1]
        \draw [#2] (-2., -1) arc (-90:90:1) --
        (-2, 2.5) --
        (2, 2.5) --
        (2, 1) arc (90:270:1) -- 
        (2, -2.5) -- 
        (-2, -2.5) -- 
        cycle ;
    \end{tikzpicture}
}
