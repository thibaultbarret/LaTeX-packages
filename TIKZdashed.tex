\tikzset{
    cheating dash/.code args={on #1 off #2 ends #3}{%
        % mostly borrowed from V5 in answer to https://tex.stackexchange.com/questions/133271/can-tikz-dashed-lines-emulate-pstricks-dashed-lines
        \csname tikz@addoption\endcsname{%Use csname so catcode of @ doesn't have do be changed.
            \pgfgetpath\currentpath%
            \pgfprocessround{\currentpath}{\currentpath}%
            \csname pgf@decorate@parsesoftpath\endcsname{\currentpath}{\currentpath}%
            \pgfmathparse{max(#1-#3,0)}\let\dashphase=\pgfmathresult%
            \pgfmathparse{\csname pgf@decorate@totalpathlength\endcsname-#1+2*\dashphase}\let\rest=\pgfmathresult%
            \pgfmathparse{#1+#2}\let\onoff=\pgfmathresult%
            \pgfmathparse{max(floor(\rest/\onoff), 1)}\let\nfullonoff=\pgfmathresult%
            \pgfmathparse{max((\rest-\onoff*\nfullonoff)/\nfullonoff+#2, #2)}\let\offexpand=\pgfmathresult%
        \pgfsetdash{{#1}{\offexpand}}{\dashphase pt}}%
    },
    cheating dash per segment/.style args={on #1 off #2 ends #3}{
        /utils/exec=\csname tikz@options\endcsname,%inherit options/.code={[\csname tikz@options\endcsname]},inherit options,
        decoration={show path construction,
                        %moveto code={},
            lineto code={\draw [cheating dash=on #1 off #2 ends #3] (\tikzinputsegmentfirst) -- (\tikzinputsegmentlast);},
            curveto code={\draw [cheating dash=on #1 off #2 ends #3] (\tikzinputsegmentfirst) .. controls (\tikzinputsegmentsupporta) and (\tikzinputsegmentsupportb) .. (\tikzinputsegmentlast);},
            closepath code={\draw [cheating dash=on #1 off #2 ends #3] (\tikzinputsegmentfirst) -- (\tikzinputsegmentlast);}
        },
        decorate,
    },
}
